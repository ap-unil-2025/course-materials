% Advanced Programming 2025 - Project Report Template
% HEC Lausanne / UNIL
\documentclass[11pt,a4paper]{article}

% Packages
\usepackage[utf8]{inputenc}
\usepackage[T1]{fontenc}
\usepackage[english]{babel}
\usepackage{amsmath,amssymb,amsthm}
\usepackage{graphicx}
\usepackage{xcolor}
\usepackage{listings}
\usepackage{hyperref}
\usepackage[margin=1in]{geometry}
\usepackage{fancyhdr}
\usepackage{float}
\usepackage{caption}
\usepackage{subcaption}
\usepackage{biblatex}
\addbibresource{references.bib} % Create this file for your references

% Code listing settings
\definecolor{codegreen}{rgb}{0,0.6,0}
\definecolor{codegray}{rgb}{0.5,0.5,0.5}
\definecolor{codepurple}{rgb}{0.58,0,0.82}
\definecolor{backcolour}{rgb}{0.95,0.95,0.92}

\lstdefinestyle{pythonstyle}{
    backgroundcolor=\color{backcolour},   
    commentstyle=\color{codegreen},
    keywordstyle=\color{magenta},
    numberstyle=\tiny\color{codegray},
    stringstyle=\color{codepurple},
    basicstyle=\ttfamily\footnotesize,
    breakatwhitespace=false,         
    breaklines=true,                 
    captionpos=b,                    
    keepspaces=true,                 
    numbers=left,                    
    numbersep=5pt,                  
    showspaces=false,                
    showstringspaces=false,
    showtabs=false,                  
    tabsize=2,
    language=Python
}

\lstset{style=pythonstyle}

% Header and footer
\pagestyle{fancy}
\fancyhf{}
\rhead{Advanced Programming 2025}
\lhead{Project Report}
\rfoot{Page \thepage}

% Title page information - MODIFY THESE
\title{%
    \Large \textbf{Advanced Programming 2025} \\
    \vspace{0.5cm}
    \LARGE \textbf{Your Project Title Here} \\
    \vspace{0.3cm}
    \large Final Project Report
}
\author{
    Your Name \\
    \texttt{your.email@unil.ch} \\
    Student ID: 12345678
}
\date{\today}

\begin{document}

\maketitle
\thispagestyle{empty}

\begin{abstract}
\noindent
Provide a concise summary (150-200 words) of your project. Include:
\begin{itemize}
    \item The problem you're solving
    \item Your approach/methodology
    \item Key results/findings
    \item Main contributions
\end{itemize}
\end{abstract}

\vspace{0.5cm}
\noindent\textbf{Keywords:} data science, Python, machine learning, [add your keywords]

\newpage
\tableofcontents
\newpage

% ================== MAIN CONTENT ==================

\section{Introduction}
\label{sec:introduction}

Introduce your project and its context. This section should include:
\begin{itemize}
    \item Background and motivation
    \item Problem statement
    \item Objectives and goals
    \item Report organization
\end{itemize}

\section{Literature Review / Related Work}
\label{sec:literature}

Discuss relevant prior work, existing solutions, or theoretical background. For data science projects, this might include:
\begin{itemize}
    \item Previous approaches to similar problems
    \item Relevant algorithms or methodologies
    \item Datasets used in related studies
\end{itemize}

\section{Methodology}
\label{sec:methodology}

\subsection{Data Description}
Describe your dataset(s):
\begin{itemize}
    \item Source and collection method
    \item Size and characteristics
    \item Features/variables
    \item Data quality issues
\end{itemize}

\subsection{Approach}
Detail your technical approach:
\begin{itemize}
    \item Algorithms used
    \item Data preprocessing steps
    \item Model architecture (if applicable)
    \item Evaluation metrics
\end{itemize}

\subsection{Implementation}
Discuss the implementation details:
\begin{itemize}
    \item Programming languages and libraries
    \item System architecture
    \item Key code components
\end{itemize}

Example code snippet:
\begin{lstlisting}[caption={Example data preprocessing function}]
def preprocess_data(df):
    """
    Preprocess the input dataframe.
    
    Args:
        df: Input pandas DataFrame
    
    Returns:
        Preprocessed DataFrame
    """
    # Remove missing values
    df = df.dropna()
    
    # Normalize numerical features
    scaler = StandardScaler()
    df[numerical_cols] = scaler.fit_transform(df[numerical_cols])
    
    return df
\end{lstlisting}

\section{Results}
\label{sec:results}

Present your findings with appropriate visualizations and tables.

\subsection{Experimental Setup}
Describe your experimental environment:
\begin{itemize}
    \item Hardware specifications
    \item Software versions
    \item Hyperparameters
\end{itemize}

\subsection{Performance Evaluation}

\begin{table}[H]
\centering
\caption{Model Performance Metrics}
\label{tab:performance}
\begin{tabular}{|l|c|c|c|}
\hline
\textbf{Model} & \textbf{Accuracy} & \textbf{Precision} & \textbf{Recall} \\
\hline
Baseline & 0.75 & 0.72 & 0.78 \\
Your Model & 0.85 & 0.83 & 0.87 \\
\hline
\end{tabular}
\end{table}

\subsection{Visualizations}

Include relevant plots and figures. For example:

\begin{figure}[H]
\centering
% \includegraphics[width=0.8\textwidth]{figures/results_plot.png}
\caption{Your results visualization}
\label{fig:results}
\end{figure}

\section{Discussion}
\label{sec:discussion}

Analyze and interpret your results:
\begin{itemize}
    \item What worked well?
    \item What were the challenges?
    \item How do your results compare to expectations?
    \item Limitations of your approach
\end{itemize}

\section{Conclusion and Future Work}
\label{sec:conclusion}

\subsection{Summary}
Summarize your key findings and contributions.

\subsection{Future Directions}
Suggest potential improvements or extensions:
\begin{itemize}
    \item Methodological improvements
    \item Additional experiments
    \item Real-world applications
\end{itemize}

% ================== REFERENCES ==================
\newpage
\section*{References}
\addcontentsline{toc}{section}{References}

% If using biblatex (recommended)
% \printbibliography[heading=none]

% Or manually:
\begin{enumerate}
    \item Author, A. (2024). \textit{Title of Article}. Journal Name, 10(2), 123-145.
    \item Smith, B. \& Jones, C. (2023). \textit{Book Title}. Publisher.
    \item Dataset Source. (2024). Dataset Name. Available at: \url{https://example.com}
\end{enumerate}

% ================== APPENDICES ==================
\newpage
\appendix
\section{Additional Figures}
\label{app:figures}

Include supplementary figures or tables that support but aren't essential to the main narrative.

\section{Code Repository}
\label{app:code}

\noindent
\textbf{GitHub Repository:} \url{https://github.com/yourusername/project-repo}

\noindent
Provide information about:
\begin{itemize}
    \item Repository structure
    \item Installation instructions
    \item How to reproduce results
\end{itemize}

\end{document}